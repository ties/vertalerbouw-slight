\chapter{Java hulpklassen}
Alle zelf ge\"{i}mplementeerde Java-klassen welke zijn gebruikt bij de Example-compiler zijn bijgevoegd in deze Appendix.

\section{Symboltable}
\subsection{Env.java}
\begin{lstlisting}[language=Java]
package edu.utwente.vb.symbols;

import java.util.Arrays;
import java.util.Collections;
import java.util.List;
import java.util.Map;
import java.util.Set;

import org.antlr.runtime.Token;
import org.antlr.runtime.tree.BaseTree;
import org.slf4j.Logger;
import org.slf4j.LoggerFactory;

import com.google.common.base.Objects;
import com.google.common.collect.HashMultimap;
import com.google.common.collect.HashMultiset;
import com.google.common.collect.ImmutableList;
import com.google.common.collect.ImmutableSet;
import com.google.common.collect.Lists;
import com.google.common.collect.Maps;
import com.google.common.collect.Multimap;
import com.google.common.collect.Multimaps;
import com.google.common.collect.SetMultimap;
import com.google.common.collect.Sets;

import edu.utwente.vb.exceptions.IllegalFunctionDefinitionException;
import edu.utwente.vb.exceptions.IllegalVariableDefinitionException;
import edu.utwente.vb.exceptions.SymbolTableException;

import static com.google.common.base.Preconditions.checkNotNull;
import static com.google.common.base.Preconditions.checkArgument;
/**
 * A Symbol Table in the style of Aho, Sethi and Ullman - Compilers: Principles, 
 * Techniques, and Tools [pg. 85, 970]
 * @author Ties
 * @param T the custom Token type
 */
public class Env<T extends BaseTree> implements EnvApi<T>{
	private Logger log = LoggerFactory.getLogger(Env.class);
	/** The table of functions for this level */
	private final SetMultimap<String, FunctionId<T>> functions 
        = HashMultimap.create();
	/** The table of variables for this level */
	private final Map<String, VariableId<T>> variables
         = Maps.newHashMap();
	/** The previous level */
	protected final Env<T> prev;
	
	/**
	 * Construct a new Environment
	 * @param p previous level
	 */
	public Env(final Env<T> p) {
		checkNotNull(p);
		this.prev = p; 
	}
	
	public Env(){
		this.prev = null;
	}
	
	/**
	 * Put a variable into the symbol table, checking if it's name is unique at 
     * the current level
	 * @throws IllegalVariableDefinitionException 
	 */
	public void put(final VariableId<T> i) throws IllegalVariableDefinitionException{
		checkNotNull(i); checkNotNull(i.getText());
		if(variables.containsKey(i.getText())){//duplicate definition in this scope level
			throw new IllegalVariableDefinitionException("duplicate definition of variable
                 \"" + i.getText() + "\" in the current scope");
		}
		variables.put(i.getText(), i);
	}
	
	/**
	 * Put a function into the symbol table, checking if it's name is unique at the current 
     * level w/ the same type of operands
	 * @throws IllegalFunctionDefinitionException 
	 */
	public void put(final FunctionId<T> i) throws IllegalFunctionDefinitionException{
		checkNotNull(i); checkNotNull(i.getText());
		if(functions.containsKey(i.getText())){//duplicate definition in this scope level
			for(FunctionId<T> potential : functions.get(i.getText())){
				if(potential.equalsSignature(i.getText(), ExampleType.asArray(i.getTypeParameters())))
					throw new IllegalFunctionDefinitionException("Function " + i.getText() 
                        + " with signature " + i.getTypeParameters().toString() 
                        + " is already defined in the current scope");		
			}
		}
		functions.put(i.getText(), i);
	}
	
	
	
	/**
	 * Get all the occurrences of a Id matching the given identifier
	 */
	public List<Id<T>> get(final String w){
		checkNotNull(w);
		ImmutableList.Builder<Id<T>> mappings = ImmutableList.builder();
		for(Env e = this; e != null; e = e.prev){
			mappings.addAll(e.functions.get(w));
			if(e.variables.containsKey(w))
				mappings.add((VariableId<T>)e.variables.get(w));
		}
		return mappings.build();
	}
	
	/**
	 * Get the binding occurrence of this applied occurrence
	 * @throws SymbolTableException 
	 * @require !applied.contains(Type.VOID) || applied.size() == 1
	 * 
	 * *: Invariant: *als* er een match was die gelijk was op hetzelfde niveau, 
     * was die gevonden bij put
	 * lijkt veel op get, maar kan de Set van daar niet hergebruiken, want geen 
     * garanties over order daarnaast is dit efficienter, hij springt er eerder uit
	 */
	@Override
	public FunctionId<T> apply(final String n, final ExampleType... applied) 
        throws SymbolTableException {
		    log.debug("applyFunction {} ({})", n, Arrays.toString(applied));
		    //zie (*)
		    for(Env e = this; e != null; e = e.prev){
			    for(FunctionId<T> id : (Set<FunctionId<T>>)e.functions.get(n)){
			    	if(id.equalsSignature(n, applied))
			    		return id;
			    }
		    }
		    log.debug("Function not found for {} ({})", n, applied);
		    throw new SymbolTableException("No matching entry for " + n + " ("  
                + Arrays.toString(applied) + ")");
	}
	
	@Override
	public VariableId<T> apply(final String n) throws SymbolTableException {
		for(Env e = this; e != null; e = e.prev){
			if(e.variables.containsKey(n))
				return (VariableId<T>) e.variables.get(n);
		}
		throw new SymbolTableException("No matching entry for variable name " + n);
	}
	
}
\end{lstlisting}
\subsection{EnvApi.java}
\begin{lstlisting}[language=Java]
package edu.utwente.vb.symbols;

import java.util.List;
import java.util.Set;

import org.antlr.runtime.Token;
import org.antlr.runtime.tree.BaseTree;

import edu.utwente.vb.exceptions.IllegalFunctionDefinitionException;
import edu.utwente.vb.exceptions.IllegalVariableDefinitionException;
import edu.utwente.vb.exceptions.SymbolTableException;

/**
 * API voor een Symbol Table (laag)
 * @author ties
 *
 * @param <T> Type van de AST Nodes
 */
public interface EnvApi<T extends BaseTree> {
	/**
	 * Voeg een functie aan de Environment toe
	 * @param i
	 */
	public void put(final FunctionId<T> i) throws IllegalFunctionDefinitionException;
	
	/**
	 * 
	 */
	public void put(final VariableId<T> var) throws IllegalVariableDefinitionException;
	
	/**
	 * Haal alle mappings van de tekst van dit ID op
	 * @param w string van het label
	 * @return Set van de Id's
	 */
	public List<Id<T>> get(final String w);
	
	/**
	 * Apply a variable
	 * @param n identifier name
	 * @param applied applied parameter types
	 * @return the id
	 */
	public VariableId<T> apply(final String n) throws SymbolTableException;
	
	/**
	 * Apply a function
	 * @param n function name
	 * @param applied types of the parameters
	 * @return
	 */
	public FunctionId<T> apply(final String n, final ExampleType... applied) throws SymbolTableException;
}
\end{lstlisting}
\subsection{FunctionId.java}
\begin{lstlisting}[language=Java]
package edu.utwente.vb.symbols;

import java.util.Arrays;
import java.util.List;

import org.antlr.runtime.Token;
import org.antlr.runtime.tree.BaseTree;
import org.objectweb.asm.commons.Method;

import com.google.common.base.Function;
import com.google.common.base.Objects;
import com.google.common.collect.Collections2;
import com.google.common.collect.ImmutableList;
import com.google.common.collect.Lists;

import edu.utwente.vb.exceptions.IllegalFunctionDefinitionException;

import static com.google.common.base.Preconditions.checkArgument;
import static com.google.common.base.Preconditions.checkNotNull;

public class FunctionId<T extends BaseTree> implements Id<T>{
	private ExampleType returnType;
	private T node;
	private List<VariableId<T>> formalParameters;
	private IdType idType = IdType.FUNCTION;
	
	/**
	 * A function identifier
	 * 
	 * Function identifiers can return Type.VOID
	 * If a function identifier has no formal parameters, it's parameters will be the empty list
	 * The formal parameters can NOT contain Type.VOID
	 * 
	 * @param t Node
	 * @param r return type
	 * @param p formal parameters.
	 * @throws IllegalFunctionDefinitionException 
	 */
	public FunctionId(T t, ExampleType r, VariableId<T>... p) throws IllegalFunctionDefinitionException{
		this.node = checkNotNull(t);
		this.returnType = checkNotNull(r);
		
		formalParameters = ImmutableList.copyOf(checkNotNull(p));
		if(formalParameters.contains(ExampleType.VOID))
			throw new IllegalFunctionDefinitionException("A function argument can not have the VOID type");
	}
	
	public FunctionId(T t, ExampleType r, List<VariableId<T>> p) throws IllegalFunctionDefinitionException{
		this.node = checkNotNull(t);
		this.returnType = checkNotNull(r);
		
		formalParameters = ImmutableList.copyOf(checkNotNull(p));
		if(extractTypes(formalParameters).contains(ExampleType.VOID))
			throw new IllegalFunctionDefinitionException("A function argument can not have the VOID type");
	}
	
	public void setIdType(IdType idType) {
		this.idType = idType;
	}
	
	
	@Override
	public ExampleType getType() {
		return returnType;
	}
	
	@Override
	public T getNode() {
		return node;
	}
	
	@Override
	public String getText() {
		return node.getText();
	}
	
	@Override
	public int hashCode() {
		return Objects.hashCode(node, returnType);
	}
	
	public List<ExampleType> getTypeParameters() {
		return extractTypes(formalParameters);
	}
	
	@Override
	public edu.utwente.vb.symbols.Id.IdType getIdType() {
		return idType;
	}
	
	@Override
	public String toString() {
		return node.getText() + " (" + Objects.toStringHelper(formalParameters) + ") -> " + returnType;
	}
	
	@Override
	public boolean equals(Object obj) {
		if(obj instanceof FunctionId){
			FunctionId that = (FunctionId)obj;
			return Objects.equal(this.node, that.node) && Objects.equal(this.returnType, that.returnType) && Objects.equal(this.formalParameters, that.formalParameters);
		}
		return false;
	}
	
	
	@Override
	public boolean equalsSignature(String name, ExampleType... applied) {
		return Objects.equal(this.node.getText(), name) && Arrays.deepEquals(applied, ExampleType.asArray(extractTypes(formalParameters)));
	}

	/**
	 * Converteer deze FunctionId naar een ASM-stijl Method descriptor.
	 * @return
	 */
	public Method asMethod(){
		return new Method(getText(), returnType.toASM(), ExampleType.listToASM(extractTypes(formalParameters)));
	}
	
	/**
	 * Transformeer een Lijst van VariableId's in een lijst van de Type's die in de VariableId's zitten
	 * @param params 
	 * @return
	 */
	private static <Q extends BaseTree> List<ExampleType> extractTypes(List<VariableId<Q>> params){
		return Lists.transform(params, new Function<VariableId<Q>, ExampleType>() {
			@Override
			public ExampleType apply(VariableId<Q> input) {
				return input.getType();
			}
		});
	}
	
	/**
	 * Update het return type van deze methode 
	 */
	@Override
	public void updateType(ExampleType t) {
		checkArgument(ExampleType.UNKNOWN.equals(this.returnType), "Return type is not unknown");
		checkArgument(!ExampleType.UNKNOWN.equals(t), "Can not update to type UNKNOWN");
		this.returnType = t;
	}
}

\end{lstlisting}
\subsection{Id.java}
\begin{lstlisting}[language=Java]
package edu.utwente.vb.symbols;

import java.util.List;

import org.antlr.runtime.Token;
import org.antlr.runtime.tree.BaseTree;

import com.google.common.collect.Lists;

public interface Id<T extends BaseTree> {
	public enum IdType{VARIABLE, FUNCTION, BUILTIN, VARARGS}
	
	public IdType getIdType();
	
	public ExampleType getType();
	
	public T getNode();
	
	public String getText();
	
	@Override
	public int hashCode();
	
	@Override
	public boolean equals(Object obj);
	
	public boolean equalsSignature(String name, ExampleType... params);
	
	public void updateType(ExampleType t);
}

\end{lstlisting}
\subsection{Prelude.java}
\begin{lstlisting}[language=Java]
package edu.utwente.vb.symbols;

import java.util.List;
import java.util.Set;

import org.antlr.runtime.CommonToken;

import com.google.common.collect.ImmutableSet;
import com.google.common.collect.Lists;
import com.google.common.collect.Sets;

import static edu.utwente.vb.symbols.ExampleType.*;
import static edu.utwente.vb.example.util.CheckerHelper.*;

import edu.utwente.vb.exceptions.IllegalFunctionDefinitionException;
import edu.utwente.vb.tree.FunctionNode;
import edu.utwente.vb.tree.TypedNode;

public class Prelude {
	public final Set<FunctionId<TypedNode>> builtins;
	
	@SuppressWarnings("unchecked")
	public Prelude() throws IllegalFunctionDefinitionException{
		ImmutableSet.Builder<FunctionId<TypedNode>> builder = ImmutableSet.builder();
		//Voeg operatoren toe
		// apply/becomes varianten
		builder.add(createBuiltin("=", BOOL, BOOL, BOOL));
		builder.add(createBuiltin("=", INT, INT, INT));
		builder.add(createBuiltin("=", CHAR, CHAR, CHAR));
		builder.add(createBuiltin("=", STRING, STRING, STRING));
		// boolean
		builder.add(createBuiltin("==", BOOL, BOOL, BOOL));
		builder.add(createBuiltin("!=", BOOL, BOOL, BOOL));
		builder.add(createBuiltin("and", BOOL, BOOL, BOOL));
		builder.add(createBuiltin("or", BOOL, BOOL, BOOL));
		builder.add(createFunctionId("!", ExampleType.BOOL, createVariableId("rhs", ExampleType.BOOL)));
		// int
		builder.add(createBuiltin("+", INT, INT, INT));
		builder.add(createBuiltin("-", INT, INT, INT));
		builder.add(createBuiltin("*", INT, INT, INT));
		builder.add(createBuiltin("/", INT, INT, INT));
		builder.add(createBuiltin("%", INT, INT, INT));
		// boolean + int
		builder.add(createBuiltin("==", BOOL, INT, INT));
		builder.add(createBuiltin("!=", BOOL, INT, INT));
		builder.add(createBuiltin("<", BOOL, INT, INT));
		builder.add(createBuiltin("<=", BOOL, INT, INT));
		builder.add(createBuiltin(">", BOOL, INT, INT));
		builder.add(createBuiltin(">=", BOOL, INT, INT));
		// string -> tijdelijk met +..
		builder.add(createBuiltin("+", STRING, STRING, STRING));
		builder.add(createBuiltin("+", STRING, STRING, CHAR));
		builder.add(createBuiltin("+", STRING, STRING, INT));
		builder.add(createBuiltin("+", STRING, STRING, BOOL));
		builder.add(createBuiltin("==", BOOL, STRING, STRING));
		builder.add(createBuiltin("!=", BOOL, STRING, STRING));
		// char
		builder.add(createBuiltin("==", BOOL, CHAR, CHAR));
		builder.add(createBuiltin("!=", BOOL, CHAR, CHAR));
		builder.add(createBuiltin("<", BOOL, CHAR, CHAR));
		builder.add(createBuiltin("<=", BOOL, CHAR, CHAR));
		builder.add(createBuiltin(">", BOOL, CHAR, CHAR));
		builder.add(createBuiltin(">=", BOOL, CHAR, CHAR));
		//Builting functions
		builder.add(createFunctionId("print", ExampleType.VOID, 	createVariableId("str", ExampleType.STRING)));
		builder.add(createFunctionId("print", ExampleType.VOID, 	createVariableId("str", ExampleType.BOOL)));
		builder.add(createFunctionId("print", ExampleType.VOID, 	createVariableId("str", ExampleType.INT)));
		builder.add(createFunctionId("print", ExampleType.VOID, 	createVariableId("str", ExampleType.CHAR)));
		
		FunctionNode readNode = byToken("read", ExampleType.VOID);
		VarArgsFunctionId<TypedNode> varArgs = new VarArgsFunctionId<TypedNode>(readNode, ExampleType.VOID, ExampleType.CHAR, ExampleType.INT, ExampleType.STRING);
		readNode.setBoundMethod(varArgs);
		builder.add(varArgs);
		
		builder.add(createFunctionId("ensure", ExampleType.VOID,	createVariableId("expr", ExampleType.BOOL)));
		//Sla op
		builtins = builder.build();
	}
	
	public void inject(SymbolTable<TypedNode> t) throws IllegalFunctionDefinitionException{
		for(FunctionId<TypedNode> node : builtins){
			t.put(node);
		}
	}
}

\end{lstlisting}
\subsection{VarArgsFunctionId.java}
\begin{lstlisting}[language=Java]
package edu.utwente.vb.symbols;

import static com.google.common.base.Preconditions.checkNotNull;

import java.util.List;
import java.util.Set;

import org.antlr.runtime.tree.BaseTree;
import org.objectweb.asm.commons.Method;

import com.google.common.collect.ImmutableList;
import com.google.common.collect.Lists;
import com.google.common.collect.Sets;

import edu.utwente.vb.exceptions.IllegalFunctionDefinitionException;

public class VarArgsFunctionId<T extends BaseTree> extends FunctionId<T>{
	private Set<ExampleType> acceptableTypes;
	
	/**
	 * A function identifier that accept any number of arguments that match the list of allowed types.
	 * 
	 * @param t Node
	 * @param r return type
	 * @param acceptable acceptable types
	 * @throws IllegalFunctionDefinitionException 
	 */
	public VarArgsFunctionId(T t, ExampleType r, ExampleType... acceptable) throws IllegalFunctionDefinitionException{
		super(t, r);
		acceptableTypes = Sets.newHashSet(acceptable);
	}
	
	@Override
	public Method asMethod() {
		throw new UnsupportedOperationException("Can not convert varArgs to fixed args");
	}
	
	@Override
	public boolean equalsSignature(String name, ExampleType... applied) {
		return getText().equals(name) && acceptableTypes.containsAll(Lists.newArrayList(applied));
	}
	
	@Override
	public List<ExampleType> getTypeParameters() {
		throw new UnsupportedOperationException("Can not get type parameters");
	}
	
	@Override
	public void updateType(ExampleType t) {
		throw new UnsupportedOperationException("Can not update type");
	}	
	
	@Override
	public edu.utwente.vb.symbols.Id.IdType getIdType() {
		return IdType.VARARGS;
	}
}

\end{lstlisting}
\subsection{VariableId.java}
\begin{lstlisting}[language=Java]
package edu.utwente.vb.symbols;

import static com.google.common.base.Preconditions.checkNotNull;
import static com.google.common.base.Preconditions.checkArgument;

import java.util.List;

import org.antlr.runtime.Token;
import org.antlr.runtime.tree.BaseTree;

import com.google.common.base.Objects;

public class VariableId<T extends BaseTree> implements Id<T>{
	private ExampleType type;
	private final T node;
	/** Variable is a constant (ie: not assignable) */
	private boolean constant = false;
	
	public VariableId(T n, ExampleType t){
		this.type =	checkNotNull(t);
		this.node = checkNotNull(n);
	}
	
	@Override
	public String getText() {
		return node.getText();
	}
	
	@Override
	public T getNode() {
		return node;
	}
	
	public ExampleType getType() {
		return type;
	}
	
	@Override
	public edu.utwente.vb.symbols.Id.IdType getIdType() {
		return IdType.VARIABLE;
	}
	
	@Override
	public int hashCode() {
		return Objects.hashCode(node, type);
	}
	
	@Override
	public boolean equals(Object obj) {
		if(obj instanceof VariableId){
			VariableId that = (VariableId)obj;
			return Objects.equal(this.node, that.node) && Objects.equal(this.type, that.type);
		}
		return false;
	}
	
	@Override
	public String toString() {
		return Objects.toStringHelper(VariableId.class).add("name", getText()).add("type",type).add("constant", constant).toString();
	}
	
	@Override
	public boolean equalsSignature(String name, ExampleType... params) {
		return Objects.equal(this.node.getText(), name) && (params == null || params.length == 0);
	}
	
	public boolean isConstant() {
		return constant;
	}
	
	public void setConstant(boolean c){
		constant = c;
	}
	
	@Override
	public void updateType(ExampleType t) {
		checkArgument(!constant, "Trying to update the type of a constant");
		checkArgument(ExampleType.UNKNOWN.equals(this.type), "Trying to update the type of a variable which is not Type.UNKNOWN");
		checkArgument(!ExampleType.UNKNOWN.equals(t), "Can not update to Type.UNKNOWN");
		this.type = t;
	}
}

\end{lstlisting}
\subsection{SymbolTable.java}
\begin{lstlisting}[language=Java]
package edu.utwente.vb.symbols;

import static com.google.common.base.Preconditions.checkArgument;

import java.util.List;
import java.util.Set;

import org.antlr.runtime.Token;
import org.antlr.runtime.tree.BaseTree;
import org.slf4j.Logger;
import org.slf4j.LoggerFactory;

import com.google.common.collect.ImmutableList;

import edu.utwente.vb.exceptions.IllegalFunctionDefinitionException;
import edu.utwente.vb.exceptions.IllegalVariableDefinitionException;
import edu.utwente.vb.exceptions.SymbolTableException;

public class SymbolTable<T extends BaseTree> implements EnvApi<T>{
	private Logger log = LoggerFactory.getLogger(SymbolTable.class);
	private Env inner;
	private int level;
	
	public SymbolTable(){
		inner = new Env();
		level = 0;
	}
	
	public void openScope(){
		log.info("openScope() " + level + " -> " + (level + 1));
		inner = new Env(inner);		
		level++;
	}
	
	public void closeScope() throws SymbolTableException{
		log.info("closeScope() " + level + " -> " + (level - 1));
		if(level <= 0)
			throw new SymbolTableException("Can not close level 0 - unbalanced indents");
		inner = inner.prev;
		level--;
	}
	
	@Override
	public void put(VariableId<T> i) throws IllegalVariableDefinitionException {
		inner.put(i);
	}
	
	@Override
	public void put(FunctionId<T> i) throws IllegalFunctionDefinitionException {
		inner.put(i);
	}
	
	@Override
	public List<Id<T>> get(String w) {
		return inner.get(w);
	}
	
	public FunctionId<T> apply(String w, List<ExampleType> applied) throws SymbolTableException{
		return apply(w, ExampleType.asArray(applied));
	}
	
	public FunctionId<T> apply(String w, ExampleType... applied) throws SymbolTableException{
		return inner.apply(w, applied);
	}
	
	public VariableId<T> apply(String n) throws SymbolTableException{
		return inner.apply(n);
	}
	
	public int getCurrentScope() {
		return level;
	}
}
\end{lstlisting}
