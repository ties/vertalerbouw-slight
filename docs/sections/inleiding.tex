\chapter{Inleiding}
Dit betreft de veslaglegging van het eindproject van het vak vertalerbouw. Gevraagd werd een eigen programmeertaal te ontwikkelen aan de hand van de in de hoorcolleges opgedane kennis waarbij aan een aantal randvoorwaarden moet zijn voldaan. De eerste randvoorwaarde voor de zelf te ontwikkelen taal is dat deze taal LL(1) is, dat wil zeggen dat het voor analyse/executie van de taal ten alle tijden voldoende is om slechts \'{e}\'{e}n woord vooruit te kijken. De andere gestelde voorwaarde is dat er bij de ontwikkeling van de programmeertaal gebruik gemaakt wordt van de tool ANTLR, versie 3. In het kader van de gestelde opdracht zijn wij aan de slag gegaan met de ontwikkeling van de programmeertaal Example, een Python-achtige taal.\\

Dit veslag zal een uitgebreide syntactische en semantische beschrijving van deze taal omvatten, evenals een verslagleggen van de procedures en tegengekomen opstakels in het ontwikkelproces. De werking van enkele zelf ge\"{i}mplementeerde java-klassen waarvan gebruikt is gemaakt als aanvulling op de functionaliteit welke ANTLR al biedt zal ook worden besproken. 

Gedurende de ontwikkeling van de Example programmeertaal is gebruik gemaakt van een zeer uitgebreide testsuite waarbij bij elke wijziging aan de taal automatisch honderden testprogramma's zijn getest. In het hoofdstuk testen zal de werking van deze testsuite demonstreren en de testresultaten van de definitieve Example behandelen.
