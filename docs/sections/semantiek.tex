\chapter{Semantiek}
Om de semantiek van de Example taal zo gestructureerd mogelijk uit te leggen wordt onderscheid gemaakt tussen vier categorie\"{e}n syntaxregels:
\begin{enumerate}
\item Declaraties
\item Functiedefinities
\item Statements
\item Expressies
\end{enumerate}
Alle BNF-regels vallen in een van deze vier categorie\"{e}n. \'{e}\'{e}n voor \'{e}\'{e}n zullen alle categorie\"{e}n nu worden behandeld, waarbij een eenduidig beeld geschetst wordt van de betekenis van de syntaxregels.

\section{Program}
{\bf program}                     ::= (declaration | functionDef)*
Het valt in de syntax op dat een Program bestaat uit een aaneenschakeling van declaration en functionDef regels. In eerste instantie voelt dit tegennatuurlijk, hoe wordt er immers ooit iets aangeroepen als een program slechts uit definities bestaat? Dit wordt veroorzaakt door de definitie van een main methode in Example. Wil men in Example een programma schrijven welke bij uitvoer direct acties uitvoert en niet enkel functies, variabelen en constanten definieerd dan moet er in het programma een main() worden opgenomen. Een main methode wordt bij uitvoer van Example programma bij uitvoer automatisch aangeroepen.

\section{Declaraties}
Een declaratie cre\"{e}ert een variabele of constante, welke is gebonden aan de opgegeven identifier.
    \subsection{Syntax}    
        \begin{tabbing}
            {\bf declaration}         ::= \=primitive IDENTIFIER valueDeclaration\\
                                      \>| primitive IDENTIFIER\\
                                      \>| \textbf{const} primitive IDENTIFIER valueDeclaration\\
                                      \>| \textbf{var} IDENTIFIER valueDeclaration\\
                                      \>| \textbf{var} IDENTIFIER\\
                                      \>| \textbf{const} IDENTIFIER valueDeclaration\\
            \\
            {\bf valueDeclaration}    ::= \textbf{=} compoundExpression\\
        \end{tabbing}
    \subsection{Semantiek}
        \begin{itemize}
        \item De variabeledeclaratie 'primitive IDENTIFIER valueDeclaration' bindt IDENTIFIER aan een nieuw gecre\"{e}erde variabele van type primitive. De waarde van de compoundExpression van valueDeclaration wordt ge\"{e}valueerd en toegekent aan de variabele die is gebonden aan IDENTIFIER. 
        \item De variabeledeclaratie 'primitive IDENTIFIER' bindt IDENTIFIER aan een nieuw gecre\"{e}erde variabele van type primitive. Er geen waarde toegekend aan de variabele gebonden aan IDENTIFIER.
        \item De constantedeclaratie 'const primitive IDENTIFIER valueDeclaration' bindt een IDENTIFIER aan een nieuw gecre\"{e}erde constante van type primitive met de waarde welke volgt uit de evaluatie van de compoundExpression van valueDeclaration.
        \item De variabeledeclaratie 'var IDENTIFIER valueDeclaration' bindt IDENTIFIER aan een nieuw gecre\"{e}erde variabele. Als deze declaratie wordt gevolgd door een valueDeclaration dan wordt het type van de variabele gebonden aan IDENTIFIER het type van de uitvoer welke volgt uit de evaluatie van de compoundExpression van valueDeclaration en wordt de waarde van de variabele gebonden aan IDENTIFIER de waarde welke volgt uit de uitvoer van de compoundExpression van valueDelaration.
        \item De variabeledeclaratie 'var IDENTIFIER' bindt IDENTIFIER aan een nieuw gecre\"{e}erde variabele. Er wordt geen type en waarde toegekend aan de variabele gebonden aan IDENTIFIER, dit gebeurd alsnog bij de eerste BECOMES op deze variabele.
        \item De constantedeclaratie 'const IDENTIFIER valueDeclaration' bindt een IDENTIFIER aan een nieuw gecre\"{e}erde constante waarbij het type van deze contstante het type wordt van de waarde van de evaluatie van de compoundExpression van valueDeclaration en de waarde wordt de waarde van de evaluatie van valueDeclaration. 
        \end{itemize}

\section{Functiedefinities}
Een definitie van een functie cre\"{e}ert een functie met de opgegeven parameters en teruggeeftype, welke gebonden is aan de opgegeven identifier.
    \subsection{Syntax}
        \begin{tabbing}
            {\bf functionDef}                 ::= \= \textbf{def} IDENTIFIER parameterDef \textbf{->} primitive \textbf{$\colon$} closedCompoundExpression\\
                                                  \>| \textbf{def} IDENTIFIER parameterDef \textbf{$\colon$} closedCompoundExpression\\
            \\
            {\bf parameterDef}                ::= \=parameterFirstElem parameterOtherElems\\
                                                  \>| parameterFirstElem\\
                                                  \>| \textbf{nothing}\\
            \\
            {\bf parameterFirstElem}          ::= primitive IDENTIFIER\\
            \\
            {\bf parameterOtherElems}         ::= \=\textbf{,}primitive IDENTIFIER\\
                                                  \>\textbf{,}primitive IDENTIFIER parameterOtherElems\\  
        \end{tabbing}
    \subsection{Semantiek}
        \begin{itemize}
        \item De functiedeclaratie 'def IDENTIFIER parameterDef -> primitive : closedCompoundExpression' bindt de instructies van closedCompoundExpressie aan een nieuw gecre\"{e}erde functie met naam IDENTIFIER. Als parameters heeft de functie de opgegeven combinatie(s) type (primitive), naam (IDENTIFIER). Het type van de teruggeefwaarde dat wordt teruggegeven aan de aanroeper is type primitive.
        \item De functiedeclaratie 'def IDENTIFIER parameterDef: closedCompoundExpression' bindt de instructies van closedCompoundExpressie aan een nieuw gecre\"{e}erde functie met naam IDENTIFIER. Als parameters heeft de functie de opgegeven combinatie(s) type (primitive), naam (IDENTIFIER). Het type van de teruggeefwaarde dat wordt teruggegeven aan de aanroeper is van het type van de return-expressies in de closedCompoundExpression.
        \end{itemize}

\section{Statements}
Statements zijn controle stucturen die gebruikt kunnen worden om de uitvoer van instructies te beinvloeden. Twee soorten statements die zijn ge\"{i}mplementeerd in de Example programeertaal zijn het if-statement en het while-statement.
    \subsection{Syntax}
        \begin{tabbing}
        {\bf statements}                  ::= \=ifStatement\\
                                          \>| whileStatement\\
        \\
        {\bf ifStatement}                 ::= \=\textbf{if} \=\textbf{(}expression\textbf{)}$\colon$ closedCompoundExpression \textbf    {else} $\colon$ closedCompoundExpression\\
                                      \>| \textbf{if} \textbf{(}expression\textbf{)}$\colon$ closedCompoundExpression\\
                                      \>| \textbf{if} expression $\colon$ closedCompoundExpression \textbf{else} $\colon$ closedCompoundExpression\\
                                      \>| \textbf{if} expression $\colon$ closedCompoundExpression\\
        \\
        {\bf whileStatement}              ::= \=\textbf{while} \textbf{(} expression \textbf{)} $\colon$ closedCompoundExpression\\
                                      \>| \textbf{while} expression $\colon$ closedCompoundExpression\\
        \\ 
        \end{tabbing}
    \subsection{Semantiek}
        \begin{itemize}
        \item Het statement 'if (expression) $\colon$ closedCompoundExpression else $\colon$ closedCompoundExpression' evalueert de uitkomst van expression. Is deze expressie waar dan dan wordt de eerste closedCompoundExpression uitgevoerd; is deze expressie niet waar dan wordt de tweede closedCompoundExpression uitgevoerd. Expression is van type bool, daar controleert de checker op.
        \item Het statement 'if expression $\colon$ closedCompoundExpression else $\colon$ closedCompoundExpression' evalueert de uitkomst van expression. Is deze expressie waar dan dan wordt de eerste closedCompoundExpression uitgevoerd; is deze expressie niet waar dan wordt de tweede closedCompoundExpression uitgevoerd. Expression is van type bool, daar controleert de checker op.
        \item Het statement 'while (expression) $\colon$ closedCompoundExpression' evalueert de uitkomst van expression; als het True is dan wordt closedCompoundExpression uitgevoerd. Daarna wordt het while-statement nogmaals uitgevoerd. Wanneer de uitkomst van expression False is, dan is het while-statement klaar. Expression is van type bool, daar controleert de checker op.
        \item Het statement 'while expression $\colon$ closedCompoundExpression' evalueert de uitkomst van expression; als het True is dan wordt closedCompoundExpression uitgevoerd. Daarna wordt het while-statement nogmaals uitgevoerd. Wanneer de uitkomst van expression False is, dan is het while-statement klaar. Expression is van type bool, daar controleert de checker op.
        \end{itemize}

\section{Expressies}
Een expressie wordt ge\"{e}evalueerd en zal mogelijk, maar in de meeste gevallen, een waarde opleveren.
    \subsection{Syntax}
        \begin{tabbing} 
            {\bf closedCompoundExpression}    ::= \textbf{INDENT} compoundMultiExpression \textbf{DEDENT}\\
            \\ 
            {\bf compoundMultiExpression}     ::= \=compoundExpression compoundMultiExpression\\
                                                  \>| compoundExpression\\
            \\
            {\bf compoundExpression}          ::= \=expression\\
                                                  \>| \textbf{return} expression\\
                                                  \>| declaration\\
            \\ 
            {\bf expression}                  ::= \=orExpression = expression\\
                                                  \>| orExpression\\
            \\   
            {\bf orExpression}                ::= \=andExpression \textbf{or} orExpression\\
                                                  \>| andExpression\\
            \\   
            {\bf andExpression}               ::= \=equationExpression \textbf{and} andExpression\\
                                                  \>| equationExpression\\
            \\ 
            {\bf equationExpression}          ::= \=plusExpression \textbf{\textless{}=} equationExpression\\
                                                  \>| plusExpression \textbf{\textgreater{}=} equationExpression\\
                                                  \>| plusExpression \textbf{\textgreater{}} equationExpression\\
                                                  \>| plusExpression \textbf{\textless{}} equationExpression\\
                                                  \>| plusExpression \textbf{==} equationExpression\\
                                                  \>| plusExpression \textbf{!=} equationExpression\\
                                                  \>| plusExpression\\

            \\ 
            {\bf plusExpression}              ::= \=multiplyExpression \textbf{+} plusExpression\\
                                                  \>| multiplyExpression \textbf{-} plusExpression\\
                                                  \>| multiplyExpression\\
            \\
            {\bf multiplyExpression}          ::= \=unaryExpression \textbf{*} multiplyExpression\\
                                                  \>| unaryExpression \textbf{/} multiplyExpression\\
                                                  \>| unaryExpression \textbf{\%} multiplyExpression\\
            \\
            {\bf unaryExpression}             ::= \=\textbf{!} simpleExpression\\
                                                  \>| simpleExpression\\ 
            \\   
            {\bf simpleExpression}            ::= \=atom\\
                                                  \>| functionCall\\
                                                  \>| variable\\
                                                  \>| paren\\
                                                  \>| closedCompoundExpression\\
                                                  \>| statements\\
            \\
            {\bf atom}                        ::= \=DIGIT+\\
                                                  \>| \textbf{'}LETTER\textbf{'}\\
                                                  \>| LETTER+\\
                                                  \>| \textbf{True}\\
                                                  \>| \textbf{False}\\
            \\   
            {\bf functionCall}                ::= IDENTIFIER \textbf{(}params\textbf{)} \\
            \\      
            {\bf variable}                    ::= IDENTIFIER \\
            \\ 
            {\bf paren}                       ::= \textbf{(} expression \textbf{)}\\
        \end{tabbing}
    \subsection{Semantiek}
        \begin{itemize}
        \item De expressie 'INDENT compoundMultiExpression DEDENT' levert de waarde op die de return-expressies van de compoundMultiExpression opleveren.
        \item De expressie 'orExpression = expression' evalueert de waard van expression en kent deze waarde aan de variabele die in orExpression zit (orExpression is ee variabele, dit dwingt de checker af). Verder wordt de waarde opgelevert die opgeleverd wordt door de evaluatie van expression.
        \item De expressie 'andExpression or orExpression' levert een boolean waarde op. andExpression wordt ge\"{e}valueerd, orExpression wordt ge\"{e}valueerd en indien een van deze True is wordt True opgeleverd, anders wordt False opgeleverd.
        \item De expressie 'equationExpression or andExpression' levert een boolean waarde op. equationExpression wordt ge\"{e}valueerd, andExpression wordt ge\"{e}valueerd en indien beide True zijn wordt True opgeleverd, anders wordt False opgeleverd.
        \item De expressie 'plusExpression \textless{}= equationExpression' levert een boolean waarde op. plusExpression wordt ge\"{e}valueerd. equationExpression wordt ge\"{e}valueerd en indien de waarde van plusExpression kleiner is of gelijk is aan de waarde van equationExpression wordt True opgeleverd, anders wordt False opgeleverd. De evaluatie van plusExpression en die van equationExpression zijn door de checker gegarandeerd gelijktypig.
        \item De expressie 'plusExpression \textgreater{}= equationExpression' levert een boolean waarde op. plusExpression wordt ge\"{e}valueerd. equationExpression wordt ge\"{e}valueerd en indien de waarde van plusExpression groter is of gelijk is aan de waarde van equationExpression wordt True opgeleverd, anders wordt False opgeleverd. De evaluatie van plusExpression en die van equationExpression zijn door de checker gegarandeerd gelijktypig.
        \item De expressie 'plusExpression \textgreater{} equationExpression)' levert een boolean waarde op. plusExpression wordt ge\"{e}valueerd. equationExpression wordt ge\"{e}valueerd en indien de waarde van plusExpression groter is dan de waarde van equationExpression wordt True opgeleverd, anders wordt False opgeleverd. De evaluatie van plusExpression en die van equationExpression zijn door de checker gegarandeerd gelijktypig.
        \item De expressie 'plusExpression \textless{} equationExpression' levert een boolean waarde op. plusExpression wordt ge\"{e}valueerd. equationExpression wordt ge\"{e}valueerd en indien de waarde van plusExpression kleiner dan de waarde van equationExpression wordt True opgeleverd, anders wordt False opgeleverd. De evaluatie van plusExpression en die van equationExpression zijn door de checker gegarandeerd gelijktypig.
        \item De expressie 'plusExpression == equationExpression' levert een boolean waarde op. plusExpression wordt ge\"{e}valueerd. equationExpression wordt ge\"{e}valueerd en indien de waarde van plusExpression gelijk is aan de waarde van equationExpression wordt True opgeleverd, anders wordt False opgeleverd. De evaluatie van plusExpression en die van equationExpression zijn door de checker gegarandeerd gelijktypig.
        \item De expressie 'plusExpression != equationExpression' levert een boolean waarde op. plusExpression wordt ge\"{e}valueerd. equationExpression wordt ge\"{e}valueerd en indien de waarde van plusExpression kleiner is of gelijk is aan de waarde van equationExpression wordt True opgeleverd, anders wordt False opgeleverd. De evaluatie van plusExpression en die van equationExpression zijn door de checker gegarandeerd gelijktypig.
        \item De expressie 'multiplyExpression + plusExpression' levert een string, char of int waarde op. multiplyExpression wordt ge\"{e}valueerd en plusExpression wordt ge\"{e}valueerd. De opgeleverde waarde hangt af van het type van de waarde die door multiplyExpression en plusExpression worden opgeleverd, de volgende gevallen zijn hierin te onderscheiden:
            \begin{itemize}
            \item Indien multiplyExpression en plusExpression beide van het type String zijn wordt een string opgelevert welke gelijk is aan de String opgelevert door multiplyExpression direct gevolgd door de String opgelevert door plusExpression.
            \item Indien multiplyExpression en plusExpression beide van het type int zijn wordt een int opgelevert welke gelijk is aan de som opgelevert door de integerwaarde van multiplyExpression op te tellen bij de integerwaarde opgelevert door plusExpression.
            \item Indien multiplyExpression van het type string is en plusExpression van het type int is wordt een string opgelevert welke gelijk is aan de String opgelevert door multiplyExpression direct gevolgd door het cijfer opgelevert door plusExpression.
            \item Indien multiplyExpression van het type string is en plusExpression van het type char is wordt een string opgelevert welke gelijk is aan de String opgelevert door multiplyExpression direct gevolgd door het character opgelevert door plusExpression.
            \end{itemize}
        \item De expressie 'multiplyExpression - plusExpression' levert een int op. multiplyExpression wordt ge\"{e}evalueerd en plusExpression wordt ge\"{e}evalueerd en de cijferwaarde van plusExpression wordt afgetrokken van de cijferwaarde van multiplyExpression en het resultaat wordt opgeleverd.
        \item De expressie 'unaryExpression * multiplyExpression' levert een int op. unaryExpression wordt ge\"{e}evalueerd en multiplyExpression wordt ge\"{e}evalueerd en de cijferwaarde van unaryExpression wordt vermenigvuldigd van de cijferwaarde van multiplyExpression en het resultaat wordt opgeleverd.
        \item De expressie 'unaryExpression / multiplyExpression' levert een int op. unaryExpression wordt ge\"{e}evalueerd en multiplyExpression wordt ge\"{e}evalueerd en de cijferwaarde van unaryExpression wordt gedeeld door de cijferwaarde van multiplyExpression, vervolgens afgerond naar beneden afgerond naar een int en het resultaat wordt opgeleverd.
        \item De expressie 'unaryExpression \% multiplyExpression' levert een int op. unaryExpression wordt ge\"{e}evalueerd en multiplyExpression wordt ge\"{e}evalueerd en van de cijferwaarde van unaryExpression en de cijferwaarde van multiplyExpression wordt de modulo bepaald en dit resultaat wordt opgeleverd.
        \item De expressie '! simpleExpression' evalueert de boolean waarde van simpleExpression. Indien simpleExpression naar True evalueerd wordt False opgeleverd en indien simpleExpression naar False evalueerd wordt True opgeleverd.
        \item De expressie 'atom' evalueert de string, int, char, of bool waarde aan de hand van de geschreven tekst en levert deze op.
        \item De expressie 'IDENTIFIER (params)' evalueert de functieaanroep van de functie met naam IDENTIFIER onder de parameters params en levert de terugkeerwaarde van deze functie op.
        \item De expressie 'IDENTIFIER' levert de waarde van de variabele op die gebonden is aan deze IDENTIFIER.
        \item De expressie '( expression )' levert niets op, maar de evaluatie van de expressie binnen de '(' en ')' is wel van invloed op while- en ifstatements. Zie ook semantiek: statements.
        \item De expressie 'closedCompoundExpression' levert de waarde op die de return-expressies van de compoundMultiExpression opleveren.
        \item De expressie 'statement' levert niets op. Voor de semantiek van deze regel, zie semantiek: statements. 
        \end{itemize}

