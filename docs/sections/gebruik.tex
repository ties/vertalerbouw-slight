\section{Compileren en gebruik van de compiler}
De compiler wordt gecompileerd met hulp van maven. Maven2 is een dependency management en build tool voor java. De volgende commando's zijn relevant:

\begin{verbatim}
mvn clean - verwijder alle build artifacts
mvn compile - compileer en test de code
mvn assembly:assembly - compileer en unit test de code
\end{verbatim}

Na het assembly commando staat de jar van de compiler in de target map. De dependencies zitten ook in de example-compiler-...-jar-with-dependencies.jar. Omdat de dependencies hier in zitten kan je de compiler starten zonder het classpath in te stellen.

\begin{verbatim}
java -jar target\example-compiler-0.1-jar-with-dependencies.jar
\end{verbatim}

Wanneer je de compiler stat probeert hij in principe code te lezen vanaf de stdin. Handige opties zijn:
\begin{tabular*}{0.75\textwidth}{ |l | l|}
	\hline
	Argument		&	Uitleg \\
	\hline
	-debug_parser	& Zet de parser in debug mode, voor debug met antlrworks\footnote{In antlrworks moet je de target\\classes directory van de compiler op je classpath toevoegen} \\
	-debug_checker	& Zet de checker in debug mode, voor debug met antlrworks \\
	-file_input [filenaam] & lees invoer vanuit [filenaam] \\
\end{tabular*}

Grafieken maken:
ties@flight:~/Dev/workspace.studie/vertalerbouw$ java -jar target/example-compiler-0.1-jar-with-dependencies.jar -dot -no_checker -file_input src/test/resources/fibonacci.ex  > fibonacci.dot
ties@flight:~/Dev/workspace.studie/vertalerbouw$ dot -Tps fibonacci.dot -o fibonacci.ps
