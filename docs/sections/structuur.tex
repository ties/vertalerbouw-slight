\chapter{Structuur}
\section{De grote lijnen}
De Example-compiler bestaat uit een aantal lagen die tezamen zorg dragen voor de executie van de programmacode. Deze lagen zijn hieronder overzichtelijk gemaakt. 
[Hier dat visio plaatje van Ties]

Per compiler-laag zullen we nu kort indiepen op de werking en de verantwoordelijkheden van deze compiler-laag. De volgorde waarin de compiler-lagen aan het bod komen komt overeen met de volgorde van de lagen in de Example compiler. 

\section{Lexer}
De lexer specificeerd alle soorten tokens die bestaan binnen de Example programmeertaal.
\section{Parser}
\section{Checker}
De verantwoordelijkheden van de lexer zijn drieledig:
\begin{itemize}
	\item Controleren of alle applied occurrences van variabelen een binding occurence hebben op dezelfde of een lagere scope waarbij type overeenkomt.
	\item Controleren of alle aanroepen van functies een functiedefinitie op dezelfde of een lagere scope hebben waarbij zowel return type als alle argument-typen overeen komen.
	\item Het toekennen van een type aan alle noden in de AST.
\end{itemize}
\section{Code Generator}
\section{Interpreter}
Example maakt gebruik van de Java Virtual Machine. Dit is een interpreter die de JVM-code die de Code Generator heeft geproduceert kan uitvoeren en zorgt dat het systeem uiteindelijk dat doet wat de programmeur heeft gespecificeerd in zijn Example-code. 
