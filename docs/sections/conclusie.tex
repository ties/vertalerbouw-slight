\chapter{Conclusie en future work}

\section{Conclusie huidige implementatie Example}
De behoorlijke dekking die is bereikt met de testscenario's die zijn opgesteld voor de Example compiler waarborgt een bepaald betrouwbaarheidniveau welke voor een Compilerproject van enkele weken zeer behoorlijk is. Dit biedt de programmeur een redelijke betrouwbaarheid bij het werken met de Example programmeertaal. Hoewel implementatie een worsteling is geweest biedt de mogelijkheid tot expliciet gebruik van return-statements de programmeur een hoop controle over de te returnen waarde, wat zeker functioneel is. Naast de betrouwbaarheid van de taal en expliciete returnstatements is een ander sterk punt van Example de programmeertaal de automatische afleiding van typen van variabelen waarvan niet expliciet een type is opgegeven. Dit beidt de programmeur de flexibiliteit een variabele te declareren waarvan hoe op het moment van declareren nog niet precies weet wat voor type hij hierin gaat declareren. Na toekenning op deze variabele is het zaak consistentie te behouden betreft het type van deze variabele, wat sterke typering van variabelen in waarborgd.\\

Een niet onbelangrijk nadeel van Example is het geringe aantal taalconstructies die in de huidige versie van de taal zijn ge\"{i}mplemeteerd. Ongelukkigerwijs was de hoeveelheid de beschikbare tijd voor het eindproject beperkt en heeft de focus gedurende het project gelegen op het ontwikkelen van een nette en betrouwbare taal. Toch is de afwezigheid van object orri\"{e}ntatie en taalconstructies als arrays en enumeratieve datatypes een gemis die afbreuk doet aan de daadwerkelijke bruikbaarheid van de taal. Al met al overheerst de mening dat Example een programmeertaal is geworden die zeer netjes ge\"{i}mplementeerd en betrouwbaar is geworden. Terugkijkend op het project zijn wij content met de keuze om focus op betrouwbaarheid en nette implementatie te verkiezen boven focus op het implementeren van een groot aantal taalconstructies, een gebrek aan betrouwbaarheid van een programmeertaal zou immers compleet afbreuk doen aan de taalconstructies dan ge\"{i}mplementeerd waren geweest.

\section{Future Work}
Hoewel de Example programmeertaal ruim voltooid is binnen het niveau van uitwerking wat van een programmeertaal binnen het vak vertalerbouw verwacht mag worden zijn er nog vele optimalisaties en uitbreidingen die gedaan kunnen worden om Example gebruiksvriendelijker en effici\"{e}nter processor- en geheugengebruik te maken. Enkele optimalisaties en uitbreidingen die boven alle worden aanbevolen zullen in een aantal secties worden behandeld.

\subsection{Optimalisatie ongebruikte variabelen}
De Example programmeertaal staat toe dat gebruikers variabelen definieren zonder daarbij direct een waarde toe te kennen aan deze variabelen. In het geval dat een onervaren programmeur aan het werk gaat met de programmeertaal of in het geval dat er een complexere/onoverzichtelijkere applicatie wordt geschreven is het denkbaar dat de programmeur per abuis varabelen declareert die vervolgens nooit worden aangeroepen. Omdat dit ten gevolge heeft dat deze variabele gedurende de gehele executie van het programma ruimte op de stack innemen is dit vanuit een oogpunt van geheugenverbruik onwenselijk. In de codegeneratie voorbereidingsfase kan een algoritme worden ge\"{i}mplementeerd die gebruik van variabelen analyseert en de volgende soorten variabelen onderscheidt:
\begin{description}
\item[nooit gelezen en nooit aan toegekend] deze variabelen hoeven nooit in het geheugen en op de stack geladen te worden gezien deze overbodig zijn.
\item[\'{e}\'{e}nmaal gelezen en nooit aan toegekend] deze variabelen hoeven ook nooit in het geheugen en de stack te worden geladen, op de plaats waar deze wordt gelezen kan de betreffende waarde rechtstreeks op de stack worden geladen.
\item[meermaal gelezen en nooit aan toegekend] deze variabelen kunnen als constante in het geheugen worden geladen, aangezien JVM constanten sneller verwerkt dan variabelen.
\item[minstens eenmaal aan toegekend] deze variabelen moeten daadwerkelijk als variabelen worden behandeld.
\end{description}
Deze optimalisatie kan een behoorlijke winst in geheugengebruik bewerkstelligen.

\subsection{Uitbreiding met datatype array}
In het dagelijks gebruik van programmeertalen is de dataconstructie array bijna niet meer weg te denken. Een array is geheugeneffici\"{e}nt en biedt een hoop functionaliteit aan een programmeur. Een combinatie van een loop-constructie in een programmeertaal met een array biedt een programmeur mogelijkheden om vele algoritmen op elegant, overzichtelijke en effici\"{e}nte wijze te implementeren.
