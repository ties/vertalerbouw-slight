\chapter{Code Templates}
In ASM zit de mogelijkheid om automatisch de java code te genereren die een bepaald stuk bytecode genereert.
Dit doet http://asm.ow2.org/asm33/javadoc/user/org/objectweb/asm/util/ASMifierClassVisitor.html

Wij hebben onze Code Templates gemaakt met deze tool; In de directory met PrintTemplate.class krijg je met de volgende aanroep een Java bestand dat de ASM aanroepen bevat waarmee je PrintTemplate.class zou genereren.
\begin{verbatim}
	java -cp ../target/dependency/asm-all-3.3.1.jar:. \
	org.objectweb.asm.util.ASMifierClassVisitor PrintTemplate PrintTemplateAsm.java
\end{verbatim}

\section{Opcodes}
\begin{tabular*}{0.9\textwidth}{@{\extracolsep{\fill}} |l| r | p{8cm} |}
	\hline
	\textbf{Opcode}	&	\textbf{Argument}	& \textbf{Explanation}	\\
	\hline
	ALOAD		&	0			& On instance method invocation, local variable \verb+0+ is always used to pass a reference to the object on which the instance method is being invoked.	\\
	BIPUSH		&	\verb+[byte]+			& The immediate \verb+byte+ is pushed onto the operand stack. \\
	ISTORE		&	\verb+[index]+			& The index is an unsigned \verb+byte+ that must be an index into the local variable array of the current frame. \\
	RETURN		&	-			& Return \verb+void+ from method. \\
	ARETURN		&	-			& Return \verb+reference+ from method. \\
	IRETURN		&	-			& Return \verb+int+ from method. \\
	\hline
\end{tabular*}

\section{Stack protocol}
Tijdens het compileren van een programma kom je sommige operaties op de stack meerdere keren tegen. Deze operaties leggen wij hieronder uit

\subsection{Push this before arguments}
\paragraph{Operation}
Push ALOAD0 onto the stack before the two highest values. This is needed when a binary operator is implemented using a Java function instead of a binary operator. The other solution to this problem is the usage of two local variables, but in our opinion this is cleaner.

\paragraph{Used at}
\begin{itemize}
	\item{String != and ==}
	\item{String append}
\end{itemize}

\paragraph{Operand stack}
\verb+[lhs] [rhs] => [this] [lhs] [rhs]+

\paragraph{Implementation}
\begin{verbatim}
// Stack protocol
// [lhs] [rhs]
SWAP
// [rhs] [lhs]
ALOAD 0
// [rhs] [lhs]] [this]
SWAP
// [rhs] [this] [lhs]
DUP2_X1
// [this] [lhs] [rhs] [this] [lhs]
POP2
// [this] [lhs] [rhs]
\end{verbatim}

\subsection{}
