\chapter{Code Templates}
In ASM zit de mogelijkheid om automatisch de java code te genereren die een bepaald stuk bytecode genereert.
Dit doet http://asm.ow2.org/asm33/javadoc/user/org/objectweb/asm/util/ASMifierClassVisitor.html

Wij hebben onze Code Templates gemaakt met deze tool; In de directory met PrintTemplate.class krijg je met de volgende aanroep een Java bestand dat de ASM aanroepen bevat waarmee je PrintTemplate.class zou genereren.
\begin{verbatim}
	java -cp ../target/dependency/asm-all-3.3.1.jar:. \
	org.objectweb.asm.util.ASMifierClassVisitor PrintTemplate PrintTemplateAsm.java
\end{verbatim}

\section{Opcodes}
\begin{tabular*}{0.9\textwidth}{@{\extracolsep{\fill}} |l| r | p{8cm} |}
	\hline
	\textbf{Opcode}	&	\textbf{Argument}	& \textbf{Explanation}	\\
	\hline
	ALOAD		&	0			& On instance method invocation, local variable \verb+0+ is always used to pass a reference to the object on which the instance method is being invoked.	\\
	BIPUSH		&	\verb+[byte]+			& The immediate \verb+byte+ is pushed onto the operand stack. \\
	ISTORE		&	\verb+[index]+			& The index is an unsigned \verb+byte+ that must be an index into the local variable array of the current frame. \\
	RETURN		&	-			& Return \verb+void+ from method. \\
	ARETURN		&	-			& Return \verb+reference+ from method. \\
	IRETURN		&	-			& Return \verb+int+ from method. \\
	\hline
\end{tabular*}

\section{Stack protocol}
Tijdens het compileren van een programma kom je sommige operaties op de stack meerdere keren tegen. Deze operaties leggen wij hieronder uit

\subsection{Push this voor de arguments}
\label{sec:push_this_before}

\paragraph{Operation}
Push ALOAD0 op de stack voor de twee hoogste waardes die op dit moment op de stack staan. Dit kan je gebruiken wanneer een binaire operator in een java documenonto the stack before the two highest values. This is needed when a binary operator is implemented using a Java function instead of a binary operator. The other solution to this problem is the usage of two local variables, but in our opinion this is cleaner.

\paragraph{Gebruikt bij}
\begin{itemize}
	\item{String != and ==}
	\item{String append}
\end{itemize}

\paragraph{Operand stack}
\verb+[lhs] [rhs] => [this] [lhs] [rhs]+

\paragraph{Implementatie}
\begin{verbatim}
// Stack protocol
// [lhs] [rhs]
SWAP
// [rhs] [lhs]
ALOAD 0
// [rhs] [lhs]] [this]
SWAP
// [rhs] [this] [lhs]
DUP2_X1
// [this] [lhs] [rhs] [this] [lhs]
POP2
// [this] [lhs] [rhs]
\end{verbatim}

\subsection{Vergelijkingsoperatoren voor primitieve typen}
\paragraph{Operatie}
Wanneer je de expressie van een \verb+if\berb+ statement voor een primitief type doet, zijn er twee aanpakken. Je kan de expressie herschrijven en de opcode van het if statement wisselen aan de hand van de operatie, of je spreekt af dat er altijd een vaste waarde op de stack staat om te springen.
Wij hebben voor de tweede aanpak gekozen.

\textbf{Let op} wij gebruiken 0 als \verb+true+ in onze bytecode, omdat IFEQ springt wanneer het argument 0/true is.

\paragraph{Gebruikt bij}
\begin{itemize}
	\item{Vergelijkingsoperatoren voor primitieve typen}
\end{itemize}

\paragraph{Operand stack}
\verb+[lhs] [rhs] => [true|false]+

\paragraph{Implementatie}
Voorbeeld code voor \verb+1 == 1+

\begin{verbatim}
//Load constants
	ICONST_1
	ICONST_1
// Conditional jump
	IFEQ L1
	I_CONST_1
	GOTO L3
L1
	I_CONST_0
L3
\end{verbatim}

