\section{Code Templates}
In ASM zit de mogelijkheid om automatisch de java code te genereren die een bepaald stuk bytecode genereert.
Dit doet http://asm.ow2.org/asm33/javadoc/user/org/objectweb/asm/util/ASMifierClassVisitor.html

Wij hebben onze Code Templates gemaakt met deze tool; In de directory met PrintTemplate.class krijg je met de volgende aanroep een Java bestand dat de ASM aanroepen bevat waarmee je PrintTemplate.class zou genereren.
\begin{verbatim}
	java -cp ../target/dependency/asm-all-3.3.1.jar:. org.objectweb.asm.util.ASMifierClassVisitor PrintTemplate PrintTemplateAsm.java
\end{verbatim}

\subsection{Opcodes}
\begin{tabular*}{0.75 \textwidth}{|l| r | p|}
	\hline
	\textbf{Opcode}	&	\textbf{Argument}	& \textbf{Explanation}	\\
	ALOAD		&	0			& On instance method invocation, local variable \verb+0+ is always used to pass a reference to the object on which the instance method is being invoked (this in the Java programming language).	\\
	BIPUSH		&	[byte]			& The immediate \verb+byte+ is sign-extended to an \verb+int+ value. That value is pushed onto the operand stack. \\
	ISTORE		&	[index]			& The index is an unsigned \verb+byte+ that must be an index into the local variable array of the current frame. \\
	RETURN		&	-			& Return \verb+void+ from method. \\
	ARETURN		&	-			& Return \verb+reference+ from method. \\
	IRETURN		&	-			& Return \verb+int+ from method. \\
\end{tabular*}
