\chapter{Ingebouwde functies en type coercion}
Om de programmeur in staat te stellen veelvoorkomende unaire en binaire operaties te uitvoeren moet Example een aanstal wiskundige operaties standaard ondersteunen. Enkele operatoren zijn toepasbaar op verschillende (combinaties van) datatypen en waarbij haar gedrag per datatype kan verschillen. 

Hieronder is een overzicht van de standaard ondersteunde binaire operaties uit de Example programmeertaal. Tevens geven we de omzettingen aan tussen typen die Example automatisch uitvoerd.

\begin{tabular*}{0.75\textwidth}{@{\extracolsep{\fill}} |l | l | l | l |}
	\hline
		Operator	    &   Left hand side	&	Right hand side	&	Result	\\
	\hline
        =               &   bool            &   bool            &   bool\\
        =               &   char            &   char            &   char\\
		=               &   int             &   int             &   int\\
        =               &   string          &   string          &   string\\

        ==              &   bool            &   bool            &   bool\\
        ==              &   int             &   int             &   bool\\
        ==              &   char            &   char            &   bool\\
        ==              &   string          &   string          &   bool\\
       
        !=              &   bool            &   bool            &   bool\\
        !=              &   int             &   int             &   bool\\
        !=              &   char            &   char            &   bool\\
        !=              &   string          &   string          &   bool\\

		or			    &	bool	        &	bool	        &	bool\\
		and			    &	bool	        &	bool	        &	bool\\
		
        +               &   string          &   string          &   string\\
        +               &   string          &   char            &   string\\
        +               &   string          &   int             &   string\\
        +               &   string          &   bool            &   string\\
        +               &   char            &   char            &   string\\

		+			    &	int	            &	int		        & 	int\\
        -			    &	int	            &	int		        & 	int\\
		$\ast$		    &	int	            &	int	            &	int\\
		/			    &	int	            &	int	            &	int\\
		\% (modulo)	    &	int	            &	int	            &	int\\

        
		
		$<$	            &	int	            &	int	            &	bool\\
		$<$=            &	int	            &	int	            &	bool\\
		$>$	            &	int         	&	int	            &	bool\\
		$>$=            &	int	            &	int	            &	bool\\
		
		$<$	            &	char            &	char            &	bool\\
		$<$=            &	char	        &	char	        &	bool\\
		$>$	            &	char         	&	char	        &	bool\\
		$>$=            &	char	        &	char	        &	bool\\

	\hline
\end{tabular*}

Naast de bine

Naast operatoren
